\documentclass[screen, aspectratio=43]{beamer}
\usepackage[T1]{fontenc}
\usepackage[utf8]{inputenc}

% Use the NTNU-temaet for beamer 
% \usetheme[style=ntnu|simple|vertical|horizontal, 
%     language=bm|nn|en, 
%     smalltitle, 
%     city=all|trondheim|alesund|gjovik]{ntnu2017}
\usetheme[style=ntnu,language=en]{ntnu2017}

\usepackage[english]{babel}
\usepackage[style=numeric,backend=biber,natbib=false,sorting=none]{biblatex}

\title[PCW-d1]{Physical Computing Workshop: Day 3}
\subtitle{Microcontrollers, APIs, sonification, and big data}
\author[A. Xamb{\'o}]{Anna Xamb{\'o}}
\institute[NTNU]{Department of Music, NTNU}
\date{18 October 2018}
%\date{} % To have an empty date

\addbibresource{example.bib} % Add bibliography database

% Set the reference style to numeric.
% See here: http://tex.stackexchange.com/questions/68080/beamer-bibliography-icon
\setbeamertemplate{bibliography item}[text] 

% Set bibliography fonts to a small size.
\renewcommand*{\bibfont}{\footnotesize}

\begin{document}

\begin{frame}
  \titlepage
\end{frame}
%
\begin{frame}
  \frametitle{Learning Outcomes}
  \begin{itemize}
    \item Get an overview of the possibilities of interaction in physical computing applied to music.
    \item Identify the main characteristics of the Arduino board.
    \item Explore the creation of interactive systems for music using the Arduino board.    
    \item Get familiar with how web APIs work.
    \item Be able to integrate data from web APIs to an Arduino program.
    \item Explore mappings from big data to sound.
    \item Be able to adapt javascript and Arduino code to a custom-made musical instrument.     
    \item Demonstrate a custom-made musical instrument in a performance setting.
    \item Reflect on the custom-made musical instrument and performance using a blogging style.   
  \end{itemize}
\end{frame}
\begin{frame}
  \frametitle{Preparation: Reading}
        \begin{itemize}
        \item Read / skim through the following article and be ready to discuss it in class:
         \begin{itemize}
         \item Collins, N., 2010. Interaction (chapter 6). In Introduction to Computer Music. Wiley. \\
         \url{https://goo.gl/zor5gN}
         \end{itemize}    
         \end{itemize}
\end{frame}
%
\begin{frame}
  \frametitle{Preparation: What to Bring to Class?}
        \begin{itemize}
        \item Your own laptop.
        \item Headphones / earplugs.
         \end{itemize}
\end{frame}
%
\begin{frame}
  \frametitle{Preparation: What We Do Provide?}
        \begin{itemize}
        \item 7 Music Angel speakers for the performance per site.
        \item 6 Arduino Kits per site.
        \item Slides: \url{XX}.
        \item Code: \url{XX}.
        \item A handout: \url{XX}.        
         \end{itemize}
\end{frame}
%
\begin{frame}
  \frametitle{Pre-knowledge Activity: Interaction}
  Be ready to discuss topics related to interaction from the suggested reading.
\end{frame}
%
\begin{frame}
  \frametitle{Outline}
      \begin{itemize}
	\item Block I: Getting familiar with the Arduino board
	\item Block II: Basic interactive behavior activities: mappings from big data to sound
	\item Block III: Rehearsal and performance
    \end{itemize}  
\end{frame}
%
\begin{frame}
  \frametitle{Exercise 1: Arduino as an IKEA kit and ``Hello, World!''}
    \begin{itemize}
    	\item Explore the content of the Arduino experimentation kit, ideally in pairs.
	\item Follow the initial steps of the booklet: get familiar with the components and install the Arduino IDE software (page 3 of the booklet.
	\item Have a close look at the breadboard.
	\item Have a close look at the Arduino board and the types of pins.
	\item The ``Hello, World!'' in Arduino: Blinking LED exercise (pages 8--9 of the booklet).
    \end{itemize}
\end{frame}
%
\begin{frame}
  \frametitle{Exercise 2: Using a Pushbutton to Control the LED}
    \begin{itemize}
    	\item Follow the indications from the link.
	%book / booklet
    \end{itemize}
\end{frame}
%
\begin{frame}
  \frametitle{Exercise 3: Fade an LED}
    \begin{itemize}
    	\item Follow the indications from the link.
	%book arduino
    \end{itemize}
\end{frame}
%
\begin{frame}
  \frametitle{Exercise 4: Arduino 8-bit sound generation}
    \begin{itemize}
    	\item From blink to bee exercise (pages 7--9 from the [link]).
	\item Playing tones (pages 9--10 from the link).
	\item Arduino's tone library (pages 10--12 from the link).
	\item Chained melodies (pages 12--14 from the link).
	\item Converting sound to text (pages 16--17).
    	\item (Optional) Follow the indications of the Music Piezo Elements exercise (pages 18--19).
	%https://www.elektormagazine.com/files/attachment/331
    \end{itemize}
\end{frame}
%
\begin{frame}
  \frametitle{Exercise 5: Serial output from P5.js}
    \begin{itemize}
    	\item Follow the instructions from the link.
	%https://itp.nyu.edu/physcomp/labs/labs-serial-communication/lab-serial-output-from-p5-js/
    \end{itemize}
\end{frame}
%
\begin{frame}
  \frametitle{Exercise 6: Visualize live sensor data with P5.js and an Arduino}
    \begin{itemize}
    	\item Follow the instructions from the link.
	%https://publiclab.org/notes/warren/02-08-2018/visualize-live-sensor-data-with-p5js-and-an-arduino
    \end{itemize}
\end{frame}
%
%\begin{frame}
%  \frametitle{Resources: XX}
%    \begin{itemize}
%    	\item
%    \end{itemize}
%\end{frame}
%
%\begin{frame}
%  \frametitle{Other Resources: Listening}
%    \begin{itemize}
%    	\item
%    \end{itemize}
%\end{frame}
%
%\begin{frame}
%  \frametitle{References}
%  \printbibliography
%\end{frame}

\end{document}
